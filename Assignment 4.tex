\documentclass[a4paper]{article}

\input{preamble.tex}
\usepackage{graphicx}
\usepackage{tikz}

\usetikzlibrary{arrows}
\usetikzlibrary{calc}

\newcommand{\hmwkTitle}{Assignment 4}
\newcommand{\hmwkDueDate}{August 2019}
\newcommand{\hmwkClass}{Sheaves in Geometric and Logic}
\newcommand{\hmwkAuthorName}{Yiming Xu}

\usepackage{scalerel,stackengine}
\stackMath
\newcommand\rwh[1]{%
\savestack{\tmpbox}{\stretchto{%
  \scaleto{%
    \scalerel*[\widthof{\ensuremath{#1}}]{\kern-.6pt\bigwedge\kern-.6pt}%
    {\rule[-\textheight/2]{1ex}{\textheight}}%WIDTH-LIMITED BIG WEDGE
  }{\textheight}% 
}{0.5ex}}%
\stackon[1pt]{#1}{\tmpbox}%
}
\parskip 1ex

\DeclareMathOperator{\Sets}{\mathbf {Sets}}
\DeclareMathOperator{\C}{\mathbf {C}}
\DeclareMathOperator{\N}{\mathbf {N}}
\DeclareMathOperator{\op}{op}
\DeclareMathOperator{\BG}{\mathbf BG}
\DeclareMathOperator{\BM}{\mathbf BM}
\DeclareMathOperator{\y}{\mathbf y}
\DeclareMathOperator{\s}{\mathbf s}
\DeclareMathOperator{\rr}{\mathbf r}
\DeclareMathOperator{\FinSets}{\mathbf {FinSets}}
\DeclareMathOperator{\Nat}{\text {Nat}}
\DeclareMathOperator{\Hom}{\text {Hom}}
\DeclareMathOperator{\Sub}{\text {Sub}}
\DeclareMathOperator{\Aut}{\text {Aut}}
\DeclareMathOperator{\Sh}{\text {Sh}}
\DeclareMathOperator{\Et}{\text {Etale}}
\DeclareMathOperator{\Cn}{\mathbb C}
\DeclareMathOperator{\B}{\mathcal B}
\DeclareMathOperator{\E}{\mathcal E}
\DeclareMathOperator{\F}{\mathcal F}
\DeclareMathOperator{\Ps}{\mathbb P}
\DeclareMathOperator{\An}{\mathbb A}
\DeclareMathOperator{\abf}{\mathbf a}
\DeclareMathOperator{\Gm}{\mathbb G_m}
\DeclareMathOperator{\mcO}{\mathcal {O}}
\DeclareMathOperator{\CHOICE}{\sf CHOICE}
\DeclareMathOperator{\pr}{\sf pr}
\DeclareMathOperator{\dom}{\text {dom}}
\DeclareMathOperator{\cod}{\text {cod}}
\DeclareMathOperator{\im}{\text {im}}
\DeclareMathOperator{\colim}{\text {colim}}
\DeclareMathOperator{\eqlz}{\text {eq}}
\DeclareMathOperator{\coeq}{\text {coeq}}
\DeclareMathOperator{\ev}{\text {ev}}

\title{\hmwkTitle}
\author{\textbf{\hmwkAuthorName}}
\date{\hmwkDueDate}

\begin{document}
\begin{titlepage}
    \maketitle
\end{titlepage}
\begin{question}
    Let $j$ be a Lawvere-Tierney topology on an elementary topos $\E$, with a corresponding natural closure operator on subobject lattice $A\mapsto \overline{A}$, and with an associated sheaf functor $\abf:\E\to \Sh_j\E$ left adjoint to the inclusion $i:\Sh_j\E\to \E$.
\end{question}

(a) Prove that if $u:A\to B$ is a $j$-dense mono in $\E$, then so is $1\times u:E\times A\to E\times B$ for any object $E$ of $\E$.
\begin{proof}
    Given $u: A\rightarrowtail B$ is a $j$-dense mono, then $\overline{A}= B$. To prove $1\times u:E\times A\to E\times B$ is a $j$-dense mono, we need to show $\overline{E\times A} = E\times B$. 

    Consider the projection map $\pi:E\times A\to A$, we have $\overline{E\times A} = \overline{\pi^{-1}(A)}$. By page 221 of SGL, $\overline{\pi^{-1}(A)}= \pi^{-1}(\overline{A})=\pi^{-1}(B)=E\times B$, as desired. 
\end{proof}

(b) Deduce from (a) that an object $E$ of $\E$ is a $j$-sheaf iff for any $j$-dense mono $u:A\to B$, the map $E^u:E^B\to E^A$ is an isomorphism.
\begin{proof}
    Using generalized elements, it suffices to show that for any object $X$ of $\E$, the map ${E^u}^*:\Hom_{\E}(X,\E^B)\to \Hom_{\E}(X,E^A)$ induces by $E^u$ is an isomorphism. By tensor-hom adjunction, it suffices to show that for any $X$ in $\E$, $(1_X\times u)^*:\Hom_{\E}(X\times B,E)\to \Hom_{\E}(X\times A,E)$ is an isomorphism. As $E$ is a $j$-sheaf, by definition of $j$-sheaf on page 223 on SGL, it suffices to show that the mono $1_X\times u$ is $j$-dense, the result follows from (a).
\end{proof}

(c) Prove that for objects $E$ and $F$ of $\E$, if $F$ is a $j$-sheaf, then the unit $\eta:E\to i\abf(E)=\abf(E)$ induces an isomorphism $F^{i\abf(E)}\overset{\cong}\to  F^E$.
\begin{proof}
    Claim: if an object $F$ of $\E$ is a sheaf, then for any object $B$, $F^B$ is again a sheaf.

    Proof: Consider a $j$-dense mono $A\rightarrowtail E$, we want an isomorphism $\Hom_{\E}(E,F^B)\cong \Hom_{\E}(A,F^B)$. By tensor-hom adjunction, it amounts to show that $\Hom_{\E}(B,F^E)\cong \Hom_{\E}(B,F^A)$ is an isomorphism,  the result follows from (b).

    Using generalized elements and tensor-hom adjunction, it suffices to prove for any object $X$ of $\E$, $\Hom_{\E}(X\times \abf (X),F)\cong \Hom_{\E}(X\times E,F)$.

    As $\abf$ is the left adjoint of the inclusion functor, for every object $A$ of $\E$ and sheaf $F$, we have $\Hom_{\Sh_j\E}(\abf(A),F)=\Hom_{\E}(\abf(A),F)\cong \Hom_{\E}(A,F)$.

    Start from $\Hom_{\E}(X\times E,F)$, we have $\Hom_{\E}(X\times E,F)= \Hom_{\E}(\abf(X\times E),F)$ by above, as $\abf$ preserves limits, as in SGL page 231, the right hand side is equal to $ \Hom_{\E}(\abf(X)\times\abf (E),F)$. By tensor-hom adjunction again, right hand side is $ \Hom_{\E}(\abf(X),F^{\abf (E)})$. By adjunction of $\abf$ again, as $F^{\abf(E)}$ is a sheaf by the claim proved above, the right hand side is $ \Hom_{\E}(X,F^{\abf (E)})\cong  \Hom_{\E}(X\times \abf (E),F)$, as desired.
\end{proof}

(d) Prove that for object $E$ of $\E$ and any subobject $A\in\Sub(E)$, the closure $\overline{A}$ can be constructed as a pullback.
\begin{center}
    \begin{tikzcd}
        \overline{A}\ar[r]\ar[d] & \abf(A)\ar[d]\\
        E\ar[r] & \abf(E)
    \end{tikzcd}
\end{center}
\begin{proof}
    Claim : For any object $A$ of $\E$, we have $\abf(A)=\abf(\overline{A})$.

    Proof: It suffices to check the universal property. We prove for any sheaf $F$, $\Hom_{\Sh_j\E}(\abf(\overline{A}),F) = \Hom_{\E}(A,F)$. By adjunction of $\abf$, left hand side is $\Hom_{\E}(\overline{A},F)$. As $F$ is a sheaf and the mono $A\rightarrowtail \overline{A}$ is $j$-dense, the result follows.

    Thus the problem reduces to prove:

    \begin{center}
        \begin{tikzcd}
            \overline{A}\ar[r]\ar[d] & \abf(\overline{A})\ar[d]\\
            E\ar[r] & \abf(E)
        \end{tikzcd}
    \end{center}

    is a pullback.

    As $\Omega_j$ classifies closed subobjects, we have:
    \begin{center}
        \begin{tikzcd}
            \overline{A}\ar[r]\ar[d] & 1\ar[d]\\
            E\ar[r] & \Omega_j
        \end{tikzcd}
    \end{center}

    is a pullback.

    As $\abf$ preserves limits, it preserves pullback and terminal object, hence $\abf(1)=1$. And by page 225 of SGL, $\Omega_j$ is a sheaf, so $\abf(\Omega_j)=\Omega_j$. Hence the diagram:

    \begin{center}
        \begin{tikzcd}
            \abf(\overline{A})\ar[r]\ar[d] & \abf(1)=1\ar[d]\\
            \abf(E)\ar[r] & \abf(\Omega_j)=\Omega_j
        \end{tikzcd}
    \end{center}

    is a pullback. Pasting it to the first square gives:

    \begin{center}
        \begin{tikzcd}
            \overline{A}\ar[r]\ar[d] & \abf(\overline{A})\ar[d]\ar[r] & 1\ar[d]\\
            E\ar[r] & \abf(E)\ar[r] & \Omega_j
        \end{tikzcd}
    \end{center}

    The outside is the second square above, by 2-out-of-3 of pullback, as the outmost square and the right hand side are pullbacks, so does the left hand side. 
\end{proof}

\begin{question}
    Let $G$ be a group object in a topos $\E$. Prove directly that $U:\E^G\to\E$ has a right adjoint [cf. $\S$7 (8).]
\end{question}
\begin{proof}
    We want the adjoint $U:\E^G\rightleftarrows \E:R$. 
    The functor $R:\E\to \E^G$ is defined by: 
    
    On object: $B\mapsto B^G$ (exponential with $G$ in $\E$.), where the action $\mu_{B^G}:G\times B^G$ is defined by for each generalized element $\langle g,f,g'\rangle: X\to G\times B^G\times G$, $(\mu (g,f)) (g')=f(m(g,g'))$, where $m$ is the multiplication $G\times G\to G$. In set-theoretic notation, we can write $(g\cdot f)(g')=f(gg')$, where for generalized elements $\langle g,f\rangle:X\to G\times B^G$, $g\cdot f$ denotes the generalized element obtained by acting $g$ on $f$, which is the composition:
    \begin{center} 
        \begin{tikzcd}
            X\ar[r,"{\langle g,f\rangle}"] & G\times B^G\ar[r,"\mu_{B^G}"] & B^G
        \end{tikzcd}
    \end{center}

    So for generalized element $\langle g,f,g'\rangle: X\to G\times B^G\times G$, $f(gg')$ is the notation for the composition:
    \begin{center}
        \begin{tikzcd}
            X\ar[r,"{\langle g,f,g'\rangle}"] & G\times B^G\times G\ar[r,"\text{move the first $G$ to the last $G$}"] & B^G\times G\times G\ar[r,"1_{B^G}\times m"] & B^G\times G\ar[r,"\ev_B"] & B
        \end{tikzcd}
    \end{center}
    More formally, $\mu_{B^G}$ is in the diagram:

    \begin{center}
        \begin{tikzcd}
        G\times B^G\times G\ar[r,"\text{move $G$}"]\ar[dd,"\mu_{B^G}\times 1_G"] & B^G\times G\times G\ar[d,"1_{B^G}\times m"]\\
        & B^G\times G\ar[d,"\ev_B"] \\
        B^G\times G\ar[r,"\ev_B"] & B
        \end{tikzcd}
    \end{center}

    The above does define a $G$-action by the associativity of multiplication of $G$.



    On morphism: $h:B\to C$ is sent to the map $B^G\to C^G$ defined by for each generalized element $\langle f,g\rangle :X\to B^G\times G$, $h(f(g))$, where by this notation, we mean the composition: 
    \begin{center}
        \begin{tikzcd}
            X\ar[r,"{\langle f,g\rangle}"] & B^G\times G\ar[r,"\ev"] & B\ar[r,"h"] & C
        \end{tikzcd}
    \end{center} 

    That is, the map $h$ takes a generalized element $f(g)$ of $B$, and gives a generalized element of $C$ (from the same object), whereas the generalized element $f(g)$ itself is obtained by evaluating the map $f$, which is a generalized element of $B^G$, at the generalized element $g$ of $G$. More formally, the map $h:B\to C$ is sent to the $R(h)$ as in 
    \begin{center}
        \begin{tikzcd}
            B^G\times G\ar[dr,"\ev_B"]\ar[dd,"R(h)\times 1_G"] & & \\
            & B\ar[dr,"h"] & \\
            C^G\times G\ar[rr,"\ev_C"] & & C
        \end{tikzcd}
    \end{center}

    We check the $R$ is the right adjoint of $U$. That is, we need natural bijection $\Hom_{\E}(A,B)\cong \Hom_{\E^G}(A,B^G)$.

    $\Hom_{\E}(A,B)\to \Hom_{\E^G}(A,B^G)$: Given $h:A\to B$, send it to $h_0:A\to B^G$ which is the transpose of $\tilde{h_0}:A\times G\to B$ defined by $\tilde{h_0}(a,g)=h(g\cdot a)$. That is, $h_0$ is the map in:
    \begin{center}
        \begin{tikzcd}
            A\times G\ar[r,"h_0\times 1_G"]\ar[d,"\tau"] & B^G\times G\ar[dddl,"\ev_{B}"] \\
            G\times A\ar[d,"\mu_A"] &  \\
            A\ar[d,"h"] & \\
            B & 
        \end{tikzcd}
    \end{center}

    To check $h_0$ is indeed a $G$-map, we need to check the diagram:
    \begin{center}
        \begin{tikzcd}
            G\times A\ar[r,"\mu_A"]\ar[d,"1_G\times h_0"] & A\ar[d,"h_0"] \\
            G\times B^G\ar[r,"\mu_{B^G}"] & B^G
        \end{tikzcd}
    \end{center}

    Both path are maps $G\times A\to B^G$, so it suffices to check their transposes $G\times A \times G\to B$ are equal, that is, these two maps times $1_G$ and compose with $\ev_B$ are equal. This amounts to check the commutivity of ($\langle g,a,g'\rangle: X\to G\times A\times G$ any generalized element): 
    \begin{center}
        \begin{tikzcd}
            X\ar[dr,"{\langle g,a,g'\rangle}"] & & & \\
              & G\times A\times G\ar[r,"\mu_A\times 1_G"]\ar[d,"1_G\times h_0\times 1_G"] & A\times G\ar[d,"h_0\times 1_G"] & \\
              & G\times B^G\times G\ar[r,"\mu_{B^G}\times 1_G"] & B^G\times G\ar[dr,"\ev_B"] & \\
              & & & B
        \end{tikzcd}
    \end{center}

    Horizontal then vertical: $\langle g,a,g\rangle\mapsto \langle g\cdot a,g'\rangle \mapsto \langle h_0(g\cdot a),g'\rangle\mapsto (h_0(g\cdot a))(g')$, and $(h_0(g\cdot a))(g') = h(g'\cdot (g\cdot a))$ by definition of $h_0$.

    Vertical then horizontal: $\langle g,a,g'\rangle \mapsto \langle g,h_0(a),g'\rangle\mapsto \langle g\cdot (h_0(a)),g'\rangle\mapsto (g\cdot (h_0(a)))(g')$. By definition of action, $(g\cdot (h_0(a)))(g') = (h_0(a))(g'\cdot g)$. By definition of $h_0$, $(h_0(a))(g'\cdot g) = h((g'\cdot g)\cdot a)$. 
    
    The result follows from $(g'\cdot g)\cdot a = g'\cdot (g\cdot a)$, since $G$ acts on $A$.

    $\Hom_{\E^G}(A,B^G)\to \Hom_{\E}(A,B)$: Given $G$-map $u:A\to B^G$, we map it to $u':A\to B$ such that $u_0(a)=\tilde{u}(a,e)$ for any generalized element $a:X\to A$, and $\tilde{u}$ is the transpose of $u$. That is, $u'$ is the composition:
    \begin{center}
        \begin{tikzcd}
             A\ar[r,"\cong"] & A\times 1\ar[r,"1\times e"] & A\times G\ar[r,"\tilde{u}"] & B
        \end{tikzcd}
    \end{center} 

    Now we check the above two maps are inverses:

    Start from $u:A\to B^G$, denote its transpose as $\tilde{u}:A\times G\to B$, $u$ is sent to $u':A\to B$, which is the composition
    \begin{center}
        \begin{tikzcd}
             A\ar[r,"\cong"] & A\times 1\ar[r,"1\times e"] & A\times G\ar[r,"\tilde{u}"] & B
        \end{tikzcd}
    \end{center} 

    Then $u'$ is sent to $u_0':A\to B^G$, which is the transpost of the map $\tilde{u_0'}:A\times G\to B$, defined by $\tilde{u_0'}(a,g)=u'(g\cdot a)$.

    We want $u=u_0'$. It suffices to show $\tilde{u_0'}(a,g)=\tilde{u}(a,g)$ for all generalized element $\langle a,g\rangle: X\to A\times G$. On the LHS, we have $\tilde{u_0'}(a,g)=u'(g\cdot a)$ by definition of $u_0'$, this is the composition:

    \begin{center}
        \begin{tikzcd}
            X\ar[r,"{\langle g,a\rangle}"] & G\times A\ar[r,"\mu_A"] & A \ar[r,"\cong"] & A\times 1\ar[r,"1_A\times e"] & A\times G\ar[r,"\tilde{u}"] & B
        \end{tikzcd}
    \end{center}

    which sends $\langle g,a\rangle\mapsto g\cdot a \mapsto \langle g\cdot a,*\rangle \mapsto \langle g\cdot a,e\rangle \mapsto \tilde{u}(g\cdot a,e)$. So we need to check $\tilde{u}(g\cdot a,e)=\tilde{u}(a,g)$.

    Using the fact that $u$ is a $G$-map, the diagram commutes:

    \begin{center}
        \begin{tikzcd}
            X\ar[dr,"{\langle g,a,e\rangle}"] & & & \\
              & G\times A\times G\ar[r,"\mu_A\times 1_G"]\ar[d,"1_G\times h_0\times 1_G"] & A\times G\ar[d,"h_0\times 1_G"] & \\
              & G\times B^G\times G\ar[r,"\mu_{B^G}\times 1_G"] & B^G\times G\ar[dr,"\ev_B"] & \\
              & & & B
        \end{tikzcd}
    \end{center}

    Horizontal first gives $\langle g,a,e\rangle\mapsto \langle g\cdot a,e\rangle\mapsto u(g\cdot a),e\rangle \mapsto (u(g\cdot a))(e)=\tilde{u}(g\cdot a,e)$.

    Vertical first gives $\langle g,a,e\rangle\mapsto \langle g,u(a),e\rangle\mapsto \langle g\cdot u(a),e\rangle \mapsto (g\cdot u(a))(e)$, this is, by definition of action of $G$ on $B^G$, $(u(a))(e\cdot g)=(u(a))(g)=\tilde{u}(a,g)$.

    So the commutivity gives the desired equality.

     
    Start with a map $h:A\to B$ in $\E$, it is sent to $h_0:A\to B^G$ which is the transpose of $\tilde{h_0}:A\times G\to B$ defined by $\tilde{h_0}(a,g)=h(g\cdot a)$. Then $h_0$ is sent to $h_0':A\to B$, which is the composition:
    \begin{center}
        \begin{tikzcd}
            A\ar[r,"\cong"] & A\times 1\ar[r,"1_A\times e"] & A\times G\ar[r,"\tilde{h_0}"] & B
        \end{tikzcd}
    \end{center}

    To check $h_0'=h$, we need $h_0'(a)=h(a)$ for all $a:X\to A$. By definition of $h_0'$, $h_0'(a)=h_0(g)(e)=h(e\cdot a)=h(a)$, as desired.

    

    







    



    
\end{proof}

\begin{question}
    $[$Artin Glueing a la Wraith (1974)$]$ Let $\phi:\E\to \F$ be a functor. Recall that the comma category $\F/\phi$ has as objects pairs $(X,a:Y\to \phi X)$, where $X\in \E,Y\in \F$, and as morphisms pairs of morphisms which give commutatives squares [CWM, p. 47].
\end{question}

(a) Suppose that $\E$ and $\F$ have finite products, and that $\phi$ preserves them. Show that $\mathcal F/\phi$ is the category of coalgebras for the comonad on $\E\times F$, which has the underlying functor $G:\E\times \F\to \E\times \F, G(X,Y)=(X,Y\times \phi X)$.
\begin{proof}
    The comonad $G$ on $\E\times \F$ which we are considering about is the functor $:\E\times \F\to \E\times \F$ defined by on object, $(X,Y)\mapsto (X,Y\times \phi X)$, and on morphism, for $f:X_1\to X_2,g:Y_1\to Y_2$, the morphism $(f,g):(X_1,Y_1)\to (X_2,Y_2)$ is sent to $(f,g\times \phi f):(X_1,Y_1\times \phi X_1)\to (X_2,Y_2\times \phi X_2)$. The natural transformation $\delta:G\to G\circ G$ is defined by on object $(X,Y)$ of $\E$, $\delta_{(X,Y)}:(X,Y\times \phi X)\to (X,Y\times \phi X\times \phi X)$ is the map $(1_X,1_Y\times \Delta)$, where $\Delta:\phi X\to \phi X\times \phi X$ is the diagonal map. The natural transformation $\epsilon :G\to I$ is defined on object $(X,Y)$ of $\E$, $\epsilon_{(X,Y)}:(X,Y\times \phi X)\to (X,Y)$ is the map $(1_X,\pi_1)$, where $\pi_1$ is the projection $\pi_1:Y\times \phi X\to Y$. The triple $(G,\delta,\epsilon)$ does give a comonad, to check the commutative square, note that $G^2(X,Y)=G(X,Y\times \phi X) = (X,Y\times \phi X\times \phi X)$, so consider the square
    \begin{center}
        \begin{tikzcd}
            (X,Y\times \phi X)\ar[r,"\delta_{(X,Y)}"]\ar[d,"\delta_{(X,Y)}"] & (X,Y\times \phi X\times \phi X)\ar[d,"\delta_{(X,Y\times \phi X)}"]\\
            (X,Y\times \phi X\times \phi X)\ar[r,"G\delta_{(X,Y)}"]  & (X,Y\times \phi X\times \phi X\times \phi X)
        \end{tikzcd}
    \end{center}

    The map $\delta_{(X,Y\times \phi X)}:(X,Y\times \phi X\times \phi X)\to (X,Y\times \phi X\times \phi X\times \phi X)$ is, by definition of $\delta$, $(1_X,1_{Y\times \phi X}\times \Delta)$. And the map $G\delta_{(X,Y)}$ is $G$ applied on the map $(1_X,1_Y\times\Delta)$, which is $(1_X,1_Y\times\Delta\times \phi(1_X))= (1_X,1_Y\times\Delta\times 1_{\phi X})$, so the square above commutes.

    For the diagram:

    \begin{center}
        \begin{tikzcd}
            & (X,Y\times \phi X)\ar[dl,"="]\ar[dr,"="]\ar[d,"{(1_X,1_Y\times\Delta)}"] & \\
            (X,Y\times \phi X) & (X, Y\times \phi X\times \phi X)\ar[l,"\epsilon_{G(X,Y)}"]\ar[r,"G\epsilon_{(X,Y)}"] & (X, Y \times \phi X)
        \end{tikzcd}
    \end{center}

    By definition of $\epsilon$ and $G$, $\epsilon_{G(X,Y)}$ is the map $(1_X,\pi_{Y\times \phi X\times \phi X\to Y\times \phi X})$. $\epsilon_{(X,Y)}:(X, Y\times \phi X)\to (X,Y)$ is $(1_X,\pi_1)$, so $G\epsilon_{(X,Y)}$ is $(1_X,\pi_1\times \phi 1_X)=(1_X,\pi_1\times 1_{\phi X})$, the diagram above commutes.

    Now consider the category of coalgebras on $G$, a $G$-coalgebra is an object $(X,Y)$ of $\E\times \F$, with $k:(X,Y)\to (X,Y\times \phi X)$ such that $1_{(X,Y)}=\epsilon_{(X,Y)}\circ k:(X,Y)\to (X,Y\times \phi X)\overset{(1_X,\pi_1)}\to (X,Y)$ and $Gk\circ k=\delta_{(X,Y)}\circ k:(X,Y)\to (X,Y\times \phi X)\overset{(1_X,1_Y\times \Delta)}\to  (X,Y\times \phi X\times \phi X)$. The first clause require $k:(X,Y)\to (X,Y\times \phi X)$ to be $(1_X,\langle 1_Y, a\rangle)$ for  some map $Y\to \phi X$, and we can check any map of such form $(1_X,\langle 1_Y, a\rangle)$ can serve as such a $k$.

    A morphism $f$ from a coalgebra $k_1=(1_{X_1},\langle 1_{Y_1},a_1\rangle):(X_1,Y_1)\to (X_1,Y_1\times \phi X_1)$  to a coalgebra $k_2=(1_{X_2},\langle 1_{Y_2},a_2\rangle ):(X_2,Y_2)\to (X_2,Y_2\times \phi X_2)$ is a map $f=(f_1,f_2):(X_1,Y_1)\to (X_2,Y_2)$, such that the diagram:
    \begin{center}
        \begin{tikzcd}
            (X_1,Y_1)\ar[r,"k_1"]\ar[d,"f"] & (X_1,Y_1\times \phi X_1)\ar[d,"Gf"] \\
            (X_2,Y_2)\ar[r,"k_2"] & (X_2,Y_2\times \phi X_2)
        \end{tikzcd}
    \end{center}

    commutes. With the constrain on $k$, the commutivity condition precisely requires the square:

    \begin{center}
        \begin{tikzcd}
            Y_1\ar[r,"a_1"]\ar[d,"f_2"] & \phi X_1\ar[d,"\phi f_1"] \\
            Y_2\ar[r,"a_2"] & \phi X_2
        \end{tikzcd}
    \end{center}

    such a map is precisely a map in $\F/\phi$. This completes the proof.
\end{proof}
(b) Conclude that if $\E$ and $\F$ are topoi and $G$ is left exact, then $\F/\phi$ is again a topos. Prove that the projection $\F/\phi\to \E$ is logical.
\begin{proof}
    Finite product of topoi is again a topos. Hence from textbook page, as $\E\times \F$ is a topos, $\F/\phi$ is a topos as the category of coalgebras over the comonad $G$.

    (Alternative: Johnstone Sketch of an elephant Volume 1, Glueing construction, directly identify the limites, exponential and subobject classifier of a comma category. This also gives a more direct answer to the second subquestion of this question.)

    Under the identification of an object of $\F/\phi$ with a coalgebra on $G$, an object $Y\overset{a}\to \phi X$ for $X$ in $\E$ is the coalgebra $(X,Y)\overset{(1_X,\langle 1_Y,a\rangle)}\to (X,Y\times \phi X)$, and the projection map sending $Y\overset{a}\to \phi X$ to $X$ is the map that sending the coalgebra  $(X,Y)\overset{(1_X,\langle 1_Y,a\rangle)}\to (X,Y\times \phi X)$ to $X$. Our aim is to prove the projection is a logical morphism, that is, prove that the first component of the underlying object of the limit of a diagram of coalgebras is the limit of the the first component of the diagram, the first component of the underlying object exponential of two coalgebras is the exponential of the first component of the underlying object of these two coalgebras, etc. That is, to prove the projection is logical, we only need to keep track on the first component of the underlying object of a coalgebra.

    For limit, we consider products, the others are similar. Follow the construction on SGL page 251, here for two coalgebras $k_1:(X_1,Y_1)\to (X_1,Y_1\times \phi X_1),k_2:(X_2,Y_2)\to (X_2,Y_2\times \phi X_2)$, the underlying object of their product is the product $(X_1,Y_1)\times (X_2,Y_2)$ in $\E\times F$, by uniqueness of product, we can check the product is $(X_1\times X_2,Y_1\times Y_2)$, and the first component is $X_1\times X_2$, which is the product of $X_1$ and $X_2$ in $\E$, as desired.

    For exponential. For two coalgebras $(B,t),(C,u)$ in $\E\times \F$, as $G$ is left exact, we have an isomorphism $\phi:G(C^B\times B)\to G(C^B)\times GB$, let $\rho:G(C^B)\to (GC)^{GB}$ denote the exponential transpose of the composition $G(C^B)\times GB\overset{\phi^{-1}}\to (GC)^{GB}\overset{Ge}\to GC$, by page 253, the underlying object of ther exponential is the equalizer of the two maps in:

    \begin{center}
        \begin{tikzcd}
            G(C^B)\ar[r,"G(u^B)"]\ar[d,"\delta"] & G((GC)^B)\ar[d,"G((GC)^t)"] \\
            G(G(C^B))\ar[r,"G(\rho)"] & G((GC)^{GB})
        \end{tikzcd}
    \end{center}

    Note that our $B,C$ are in $\E\times \F$, so the diagram above consists of two separate pieces of information on each component. We are interested in the first component, but every map in the above diagram does nothing to the first component. Hence the equalizer is the whole $G(C^B)$, if the underlying objects of $B,C$ are $(X_1,Y_1),(X_2,Y_2)$ repectively, then the underlying object of $C^B$ is $(X_2^{X_1},Y_2^{Y_1})$, and the first component is $X_2^{X^1}$, as desired.

    For the subobject classifier. According to the construction on page 255, the underlying object is the equalizer of the two maps $G\Omega\overset{G\tau\circ \delta_{\Omega}}{\underset{1}\rightrightarrows} G\Omega$, where $\Omega$ is the subobject classifier in $\E\times \F$ and $\tau:G\Omega\to \Omega$ is the classifying map of $\Omega$. The object $\Omega$ here is the object $(\Omega_{\E},\Omega_{\F})$ in $\E\times \F$. Again, the two maps both does nothing to the first component. Hence the first component of the equalizer is $\Omega_{\E}$, as desired.

\end{proof}

(c) Consider the special case where $\F=\Sets$ and $\phi$ is the global sections functor $\Gamma:\Hom(1,-):\E\to \Sets$. Prove that if $\E=\Sh(X)$ for a topological space $X$, then $\Sets/\Gamma$ is again a topos of sheaves on a space $\hat{X}$, by giving an explicit description of the space $\hat{X}$.
\begin{proof}
    Reference: taken the idea in the paper `On the Freyd Cover of a Topos' by Ieke Moerdijk.

    The space $\hat{X}$ is a space $X\cup \{*\}$ where $*$ is a closed but not open point (take the basis as open sets in $X$ and the whole space $X\cup \{*\}$). 
    
    The key fact is that $\Hom_{\Sh(X)}(*,F)$ is $F(X)$!!! So $\Gamma:\Sh(X)\to \Sets$ is defined by $X\mapsto F(X)$.

    Firstly, observe that the objects in $\Sets/\Gamma$ are maps $S\to F(X)$, where $F$ is a sheaf on $X$ and $S$ is a set. 
    
    A sheaf on $\hat{X}$ is a functor $F:\mcO(\hat{X})^{\op}\to \Sets$ which satisfies the equalizer conditions, if we have a sheaf $F_0$ on $X$, then to extend it to a sheaf on $\hat{X}$, we need no more than a map $S\to F_0(X)$, where $S$ is any set. To see why it suffices, once we have such a map, we have a sheaf $F$ defined on $\hat{X}$ by $F(\hat{X})=S$, the restriction map is the given map $S\to F(X)$. Since from our topology on $\hat{X}$, the only cover of $\hat{X}$ is $\{\hat{X}\}$ (plus the fact that $F_0$ is a sheaf on $X$), the sheaf condition on $\hat{X}$ is trivially satisfied. 
    

    It suffices to define two functors $L:\Sh(\hat{X})\rightleftarrows\Sets/\Gamma : R$ and check they are inverses.

    The functor $L:\Sh(\hat{X})\to \Sets/\Gamma$ is defined by:

    On objects: Let $F$ be a sheaf on $\hat{X}$, then $L$ sends $F$ to the restriction map $F(\hat{X})\to F(X)$ induced by the inclusion $X\subseteq \hat{X}$.

    On morphisms: Given $F_1\to F_2$ map between sheaves on $\hat{X}$, we have 
    \begin{center}
        \begin{tikzcd}
            F_1(\hat{X})\ar[r]\ar[d] & F_1(X)\ar[d] \\
            F_2(\hat{X})\ar[r] & F_2(X)
        \end{tikzcd}
    \end{center}

    commutes by definition of map in the category of sheaves on $\hat{X}$. Moreover, $F_1,F_2$ are sheaves restricted to $X$.

    The inverse $R:\Sets/\Gamma \to \Sh(\hat{X})$ is defined by: Given $S\to F_0(X)$ where $F_0$ is a sheaf on $X$, send it to the sheaf $F$ on $\hat{X}$ defined by $F(X)=F_0(X)$ and $F(\hat{X})=S$, and the map $S\to F_0(X)$ provides the restriction map.

    Trivial to check that they are indeed inverses.


  
\end{proof}


\end{document}