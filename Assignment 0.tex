\documentclass[a4paper]{article}

\input{preamble.tex}
\usepackage{graphicx}
\usepackage{tikz}

\usetikzlibrary{arrows}
\usetikzlibrary{calc}

\newcommand{\hmwkTitle}{Assignment 0}
\newcommand{\hmwkDueDate}{22 March 2019}
\newcommand{\hmwkClass}{Sheaves in Geometric and Logic}
\newcommand{\hmwkAuthorName}{Yiming Xu}

\DeclareMathOperator{\Sets}{\mathbf {Sets}}
\DeclareMathOperator{\C}{\mathbf {C}}
\DeclareMathOperator{\N}{\mathbf {N}}
\DeclareMathOperator{\op}{op}
\DeclareMathOperator{\BG}{\mathbf BG}
\DeclareMathOperator{\BM}{\mathbf BM}
\DeclareMathOperator{\y}{\mathbf y}
\DeclareMathOperator{\FinSets}{\mathbf {FinSets}}
\DeclareMathOperator{\Nat}{\text {Nat}}
\DeclareMathOperator{\Hom}{\text {Hom}}
\DeclareMathOperator{\Sub}{\text {Sub}}
\DeclareMathOperator{\Cn}{\mathbb C}
\DeclareMathOperator{\Ps}{\mathbb P}
\DeclareMathOperator{\An}{\mathbb A}
\DeclareMathOperator{\Gm}{\mathbb G_m}


\title{\hmwkTitle}
\author{\textbf{\hmwkAuthorName}}
\date{\hmwkDueDate}

\begin{document}
\begin{titlepage}
    \maketitle
\end{titlepage}
\begin{question}
    Show that pullbacks of epis are epi for categories of each of the types (i)-(ix)
    \begin{proof}

        (i) $\Sets$, the category of all small sets $S,T$, and functions $S\to T$ between them.\newline
        Consider the diagram below where $X,Y, B$ are sets, by the last paragraph in page 29, the pullback is $P=\{\langle x,y\rangle\mid fx = gy\}$ and $f',g'$ are projections. 
        \begin{center}
            \begin{tikzcd}
               P=\{\langle x,y\rangle\mid fx = gy\}\arrow[r,dashed,"f'"]\arrow[d,dashed,"g'"] & Y\arrow[d,"g"]\\
                X\arrow[r,"f"] & B
            \end{tikzcd}
        \end{center}
        Epis in $\Sets$ are set-theoric surjections. So suppose $g$ is surjective, we prove $g'$ is surjective. This is proving for any $x\in X$, there exists $y\in Y$ such that $fx = gy$, this is because of the surjectiveness of $g$.
        \newline
        (ii) $\Sets \times \Sets$, the category of all pairs of sets, with morphisms pairs of functions.
        \newline
        An epi in $\Sets^2$ is a pair of epis in $\Sets$. Then the result follows from (i).\newline
        (iii) $\Sets^n$, the category of all $n$-tuples of sets with morphisms all $n$-tuples of functions. Here $n$ is a fixed natural number.\newline
        By induction using (ii).\newline
        (iv) $\BG$, or $G$-$\Sets$, the category of all representations of a fixed group $G$.\newline
        An epi $X\to Y$ in $\BG$ is just a set-theoric surjection $X\to Y$ which respects the $G$-action.
        For $X\overset{f}\rightarrow B\overset{g}\leftarrow Y$ in $\BG$, the pullback is $P=\{\langle x,y\rangle\mid fx = gy\}$ with $P\to X,P\to Y$ projection maps. The action by $G$ on $P$ is coordinatewise. So the result follows from (i).\newline
        (v) $\BM$, or $M$-$\Sets$, the category of all representations $X\times M\to X$ of a fixed monoid $M$ on a variable set $X$.\newline
        Same as (iv).\newline
        (vi) $\Sets^2$, the category whose objects are all functions $\sigma:X\to X'$ from one set $X'$, with evident arrows between these objects.\newline
        An epi in $\Sets$ from $\sigma_X:X_1\to X_2$ to $\sigma_Y:Y_1\to Y_2$ is pair of functions $f_1,f_2$ such that the diagram:

        \begin{center}
            \begin{tikzcd}
                X_1 \arrow[r, "\sigma_X"]\arrow[d,"f_1"]& X_2\arrow[d,"f_2"]\\
                Y \arrow[r,"\sigma_Y"]& Y'
            \end{tikzcd}
        \end{center}
        commutes where both $g_1,g_2$ are surjections. And for morphisms from $X_1\to X_2$, $Y_1 \to Y_2$ to $B_1\to B_2$, a pullback square looks like:
        \begin{center}
        \begin{tikzcd}
            & P_2\arrow[rr,"f'_2"]\arrow[dd,"g'_2",near end] & & Y_2\arrow[dd,"g_2"]\\
            P_1\arrow[ur,"\sigma_P"]\arrow[rr, "f'_1",crossing over]\arrow[dd,"g'_1"] & & Y_1\arrow[ur,"\sigma_Y"] & \\
            & X_2\arrow[rr,"f_2",near start] & & B_2\\
            X_1\arrow[ur,"\sigma_X"]\arrow[rr,"f_1"] & & B_1\arrow[from=uu,"g_1",crossing over]\arrow[ur,"\sigma_B"] &

        \end{tikzcd}
        \end{center}

        We can check both the forth and back squares are pullback squares in $\Sets$, the result follows by (i).\newline
        (vii) $\Sets^{\N}$, the category whose objects are all sequences $X$,
        \[X_0\to X_1\to X_2\to\cdots\]
        of sets $X_n$ and functions $X_n\to X_{n+1}$, with evident arrows $X\to Y$. \newline
        Similar to (vi), a pullback in $\Sets^{\N}$ looks like:
        \begin{center}
        \begin{tikzpicture}
            \node at (0,0) (a) {$X_1$};
            \node at (0,2) (b) {$P_1$};
            \node at (4,0) (c) {$B_1$};
            \node at (4,2) (d) {$Y_1$};
            \node at (1,3) (e) {$P_2$};
            \node at (1,1) (f) {$X_2$};
            \node at (5,1) (g) {$B_2$};
            \node at (5,3) (h) {$Y_2$};
            \node at (2,4) (i) {$P_3$};
            \node at (6,4) (j) {$Y_3$};
            \node at (2,2) (k) {$X_3$};
            \node at (6,2) (l) {$B_3$};
            \node at (3,5) (m) {$\cdots$};
            \node at (7,5) (n) {$\cdots$};
            \node at (3,3) (o) {$\cdots$};
            \node at (7,3) (p) {$\cdots$};
            \path[-stealth,commutative diagrams/.cd, every arrow, every label]
                     (a) edge (c) 
                     (b) edge (a)
                     (d) edge[commutative diagrams/crossing over] (c) 
                     (b) edge (d)
                     (e) edge (h)
                     (f) edge (g)
                     (a) edge (f)
                     (c) edge (g)
                     (b) edge (e)
                     (d) edge (h)
                     (e) edge (i)
                     (h) edge (j)
                     (g) edge (l)
                     (e) edge (f)
                     (h) edge (g)
                     (f) edge (k)
                     (k) edge (l)
                     (i) edge (k)
                     (j) edge (l)
                     (i) edge (j)
                     (i) edge (m)
                     (k) edge (o)
                     (l) edge (p)
                     (j) edge (n)
                     ;
\end{tikzpicture}
\end{center}
        
        Each face consists with $P_n,X_n,Y_n,B_n$ are pullback squares, so the result follows from (i).
    


        (viii) $\Sets^{\C^{\op}}$, where $\C$ is a fixed small category. Objects are all functors $P:\C^{\op}\to \Sets$ and arrows $P\to P'$ are all natural transformations $\theta:P\to P'$ between such functors.\newline
        An epi in $\Sets^{\C^{\op}}$ from $P$ to $P'$ is a natural transformation $\theta$ such that for any object $C$ of $\C$, $\theta_C:P(C)\to P'(C)$ is an epi. By page 30, the pullback of $X\rightarrow B\leftarrow Y$ for $X,Y,B: \C^{\op}\to\Sets$, the pullback $P:\C^{\op}\to\Sets$ is $(X\times_B Y)(C)\cong X(C)\times_{B(C)}Y(C)$. The result follows from (i).\newline
        (ix) $\Sets / J$, the comma category, with objects all sets over fixed set $J$.\newline
        By page 29, the comma category $\Sets / J$ is equivalent to the functor category $\Sets^J$, so the result follows from (viii).
    \end{proof}
\end{question}
\begin{question}
    Prove $\FinSets^{\N}$ has no subobject classifier.
    \begin{proof}
        Suppose, in order to get a contradiction, that there exists $\Omega: \N \to \FinSets\subseteq \Sets$ such that $\Omega$ is a subobject classifier of $\FinSets^{\N}$. In particular, $\Omega$ must classify the subobjects of each representable functor $\Hom_{\N^{\op}}(-,n):\N \to \FinSets$ for any object $n$ of $\N$. Therefore,
        \[\Sub_{\FinSets^{\N}}(\Hom_{\N^{\op}}(-,n))\cong \Hom_{\FinSets^{\N}}(\Hom_{\N^{\op}}(-,n),\Omega)\cong  \Hom_{\Sets^{\N}}(\Hom_{\N^{\op}}(-,n),\Omega)\cong \Omega(n)\]
        
        The first equivalence is by definition of subobject classifier, second equivalence is by the fact that $\FinSets^{\N}$ is a full subcategory of $\Sets^{\N}$, and the final equivalence is by Yoneda's lemma. So if such an $\Omega$ exist, it must be defined as $\Omega(n)=\{S \ \mid S \ \text{is a subfunctor of $\Hom_{\N^{\op}}(-,n)$}\}$ for all $n\in \N$. By page 38, the right hand side is $\{S \ \mid S \ \text{is a sieve on $n$}\}$. So to get the contradiction, it suffices to prove that the collection of sieves on some object $n\in\N$ is a infinite set.

        For any $n\in\N$, there exists infinitely many natural numbers which is not less than it, and for each number $a\ge n$, $\{b\ \mid \ b \ge a \}$ is a sieve on $n$. So for each object $n\in\N$, the sieves on $n$ is an infinite set. Hence such an subobject classifier does not exist.
    \end{proof}
\end{question}

\begin{question}
    (a) In $\BM=\Sets^{M^{\op}}$ for $M$ a monoid observe that an object $X$ is a right action $X\times M\to X$ of $M$ on a set $X$ and that, $Y$ being another object, $\Hom(X,Y)$ is the set of equivarient maps $e: X\to Y$ [maps with $e(xm)=(ex)m$ for all $x\in X,m\in M$]. Prove that the exponent $Y^X$ is the set $\Hom(M\times X,Y)$ of equivarient maps $e: M\times X\to Y$, where $M$ is the set $M$ with right action by $M$, with the action $e\mapsto ek$ of $k\in M$ on $e$ defined by $(ek)(g,x)=e(kg,x)$.\newline
    \begin{proof}
        By definition of exponential, we are proving the natural bijection $\Hom(Z\times X,Y)\cong \Hom(Z,\Hom(M\times X,Y))$. We define the bijection explicitly.\newline 
        Given a map $f\in \Hom(Z\times X,Y)$ in $\BM$, it is a function $f: Z\times X\to Y$ such that for all $z\in Z,x\in X, m\in M$, $f(z,x)\cdot m = f(zm, xm)$ $(*_1)$, the map in $\Hom(Z,\Hom(M\times X,Y))$ corresponds to it is defined by $z\mapsto ((m,x)\overset{f'}\mapsto f(zm,x))$. To check such an $f'\in \Hom(M\times X,Y)$, consider the diagram:
        \begin{center}
            \begin{tikzcd}
                (M\times X)\times M\ar[r,"f'\times 1"]\ar[d,"\mu"] & Y \times M\ar[d,"\mu"]\\
                M\times X\ar[r,"f'"] & Y
            \end{tikzcd}
        \end{center}
        For any $((m,x),a)\in (M\times X)\times M$, following the horizontal map first gives $f(zm,x)\cdot a$, which by $(*_1)$ is $f(zma,xa)$. And note that the action of $M$ on $M\times X$ is defined componentwise, so following the vertical map first gives the same result. Hence the diagram commutes.\newline
        A map $f\in \Hom(Z,\Hom(M\times X,Y))$ is a map $f:Z\to \Hom(M\times X,Y)$  such the diagram:
        \begin{center}
            \begin{tikzcd}
                Z\times M\ar[r,"f\times 1"]\ar[d,"\mu"] & \Hom(M\times X,Y)\times M\ar[d,"\mu"]\\
                Z\ar[r,"f"] & \Hom(M\times X,Y)
            \end{tikzcd}
        \end{center}
        commutes, that is, for all $z\in Z, m\in M$, $f(z)\cdot m = f(z\cdot m)$ $(*_3)$. By the definition of the action of $M$ on equivarient maps, this is saying that for all $k\in M,a\in X$, $(f(z)\cdot m) (k,a)=f(z)(mk,a)=f(z\cdot m)(k,a)$ $(*_2)$. Given such a map, it is correspond to the map $f_0: Z\times X\to Y$ defined as $(z,x)\overset{f_0}\mapsto f(z)(id_M,x)$. To check $f'\in \Hom(Z\times X,Y)$, consider the diagram:
        \begin{center}
            \begin{tikzcd}
                (Z\times X)\times M\ar[r,"f_0\times 1"]\ar[d,"\mu"] & Y\times M\ar[d,"\mu"] \\
                Z\times X\ar[r,"f_0"] & Y
            \end{tikzcd}
        \end{center}
        For any $z\in Z,x\in X,m\in M$, following the horizontal map first gives $f_0(z,m)\cdot m=(f(z)(id_M,x))\cdot m$, and following the vertical map first gives $f_0(z\cdot m,x\cdot m)= f(z\cdot m)(id_M,x\cdot m)$. But by $(*_2)$ we also have $f(z\cdot m)(id_M,x\cdot m)=f(z)(m,x\cdot m) = f(z)((id_M,x)\cdot m)$. So the diagram commutes because $(f(z)(id_M,x))\cdot m = f(z)((id_M,x)\cdot m)$ by $(*_3)$.

        It left to show that the two maps are inverses. Given $f: Z\times X\to Y$, it is sent to $f':Z\to \Hom(M\times X,Y)$ defined by $z\mapsto ((m,x)\mapsto f(z\cdot m,x)$. And this map is then sent to the map $Z\times X\to Y$ defined by $(z,x)\mapsto f'(z)(id_M,x)$, which is $f(z,x)$ by definition of $f'$. So we get the map back. Also, start with a map $f: Z\to \Hom(M\times X,Y)$, it is sent to $f_0: Z\times X\to Y$ defined by $(z,x)\mapsto f(z)(id_M,x)$, and then sent to the map $f_0': Z\to \Hom(M\times X,Y)$ defined by $z\mapsto ((m,x)\mapsto f_0(zm,id_M,x))$, which is $f(zm)(id_M,x)$ by definition of $f_0$. But also $f(zm)(id_M,x)= f(z)(m,k)$ by $(*_2)$. So the maps are inverses.


    \end{proof}
    (b) For objects $X,Y$ in $\Sets^{G^{\op}}$, for $G$ a group, show that the exponent $Y^X$ can be described as the set of all functions $f: X\to Y$, with the right action of $g\in G$ on such a function defined by $(fg)x= [f(xg^{-1})]g$ for $x\in X$.\newline
    \begin{proof}
        We are proving that for all $X,Y, Z\in \Sets^{G^{\op}}$, $\Hom(Z\times X,Y)\cong \Hom(Z,\Hom(X,Y))$.

        A map $f:\Hom(Z\times X, Y)$ is a map $Z\times X\to Y$ such that for all $z\in Z,x\in X, g\in G$, $f(z,x)\cdot g = f(z\cdot g,x\cdot g)$ $(*_1)$. Given such a map, it corresponds to the map $f'\in \Hom(Z,\Hom(X,Y))$ defined by $z\mapsto (x\mapsto f(z,x))$. To check such an $f'$ is indeed a map in $\Sets^{\op}$, consider the diagram:
        \begin{center}
            \begin{tikzcd}
                Z\times G\ar[r,"f'\times 1"]\ar[d,"\mu"] & \Hom(X,Y)\times G\ar[d,"\mu"]\\
                Z\ar[r,"f'"] & \Hom(X,Y)

            \end{tikzcd}
        \end{center}
        We should check that for all $z\in Z,g\in G$, $f'(z)\cdot g = f'(z\cdot g)$. That is, for all $x\in X$, $(f'(z)\cdot g)(x)=f'(z\cdot g)(x)$. By definition of action of $G$ on functions $X\to Y$, the left hand side is $(f'(z)(xg^{-1}))\cdot g$. And by definition of $f'$, $(f'(z)(xg^{-1}))\cdot g = f(z,xg^{-1})\cdot g\overset{(*_1)} = f(zg,x)$. Also the right hand side is $f(z\cdot g,x)$ by definition of $f'$. So the diagram above commutes.

        A map $f\in\Hom (Z, \Hom(X,Y))$ is a map $f:Z\to \Hom(X,Y)$ such that for all $g\in G$, $f(z)\cdot g = f(z\cdot g)$ $(*_2)$. Given such a map, it corresponds to a map $f_0:Z\times X\to Y$ defined by $(z,x)\mapsto f(z)(x)$. To check $f_0\in\Hom(Z\times X,Y)$, consider the diagram :
        \begin{center}
            \begin{tikzcd}
                (Z\times X)\times G\ar[r,"f_0\times 1"]\ar[d,"\mu"] & Y \times G\ar[d,"\mu"] \\
                Z\times X\ar[r,"f_0"] & Y
            \end{tikzcd}
        \end{center}
        We need $(f_0(z,x))\cdot g = f_0(z\cdot g,x \cdot g)$. By definition of $f_0$, it amounts to check $(f(z)(x))\cdot g = f(z\cdot g)(x\cdot g)$. By $(*_2)$, the right hand side is $(f(z)\cdot g)(x\cdot g)$, which equals to $[f(z)((x\cdot g)g^{-1})]\cdot g = f(z)(x)\cdot g$.

        The map we defined here is just currying and uncurrying, so it is obvious that the functions we defined above are inverses.
    \end{proof}
\end{question}

\begin{question}
    Generalize Theorem 2 of Section 9 to presheaf categories. More precisely, prove that for a morphism (i.e., a natural transformation) $f: Z\to Y$ in $\widehat{\C}=\Sets^{\C^{\op}}$, the pullback functor 
    \[f^*: \Sub_{\widehat{\C}}(Y)\to \Sub_{\widehat{\C}}(Z)\]
    has both a left adjoint $\exists_f$ and a right adjoint $\forall_f$. [Hint: the left adjoint can be constructed by taking the pointwise image. Define the right adjoint $\forall_f$ on a subfunctor $S$ of $Z$ by $\forall_f(S)(C)=\{y\in Y(C)\mid \text{for all $u: D\to C$ in $\C$ and $z\in Z(D)$, $z\in S(D)$ whenever $f_D(z)=yu$}\}$.]
    \begin{proof}
        By page 29 and 30, as $T$ is a subfunctor of $Y$, for any $C\in \C$, $f^*T(C)=f^{-1}_C(T(C))\subseteq Z(C)$.\newline
        Left adjoint: Define $\exists_f:\Sub_{\widehat{\C}}(Z)\to \Sub_{\widehat{\C}}(Y)$ by for any subfunctor $S$ of $Z:\C^{\op}\to \Sets$, $\exists_f S(C)= f_C(S(C))$ is the image of $S(C)$ in $Y(C)$ under $f$. To prove 
        \[\exists_f:\Sub_{\widehat{\C}}(Z)\rightleftarrows \Sub_{\widehat{\C}}(Y): f^*\]
        is a pair of adjoints is to prove for any subfunctor $S$ of $Z$ and subfunctor $T$ of $Y$, $\exists_f S$ is a subfunctor of $T$ iff $S$ is a subfunctor of $f^*T$. 

        Saying $S$ is a subfunctor of $f^*T$ is saying that for all $C\in \C$, $S(C)\subseteq f^*(T(C))= \{y\in Y(C)\mid f_C(y)\in T(C)\}$, and it is clear that this is equivalent to saying all $C\in \C$, $f_C(S(C))\subseteq T(C)$.\newline

        Right adjoint: Define $\forall_f$ as in the hint, proving the adjunction:
        \[f^*:\Sub_{\widehat{\C}}(Y)\rightleftarrows \Sub_{\widehat{\C}}(Z): \forall_f\]
        is to prove that for all subfunctor $S\in \Sub_{\widehat{\C}}(Z),T\in \Sub_{\widehat{\C}}(Y)$, $f^*T$ is a subfunctor of $S$ iff $T$ is a subfunctor of $\forall_fS$.
        
        
        This amounts to show that $\forall C\in \C,f^{-1}(T(C))\subseteq S(C)$ iff $\forall C\in \C, T(C)\subseteq \forall_fS(C)=\{y\in Y(C)\mid \forall u: D\to C, \forall z\in Z(D), f_D(z)=yu\implies z\in S(D)\}$. 

        Suppose $f^*T$ is a subfunctor of $S$, we prove for any $C\in \C, t\in T(C)\implies \forall u: D\to C,\forall z\in Z(D),f_D(z)=tu \implies z\in S(D)$. Fix such $C,t,u,z$. As $f^*T$ is a subfunctor of $S$, the fact that $f_D(z)=tu\in T(D)$ implies $z\in S(D)$. Conversely, suppose for any $C\in \C, t\in T(C)\implies \forall u: D\to C,\forall z\in Z(D),f_D(z)=tu \implies z\in S(D)$, let $A\in \C$, we prove $f^{-1}(T(A))=\{z\in Z(A)\mid f_A(z)\in T(A)\}\subseteq S(A)$. Let $z\in Z(A)$ such that $f_A(z)\in T(A)$, plug in $A$ for $C$, $f_A(z)$ for $t$, $1:A\to A$ for $u$, $z$ for $z$ gives gives $z\in S(A)$. 
    \end{proof}
\end{question}


\begin{question}
    Prove Proposition 5.1, that every functor $P$ to sets is representable, by constructing for each $P: \mathbf C^{\op} \to \Sets$ a coequalizer.
    $$\underset{\underset{\underset{p\in P(C)}{C'\to C}}u}\coprod \mathbf y (C')\overset{\theta}{\underset{\tau}\rightrightarrows}\underset{p\in P(C)}{\underset{C\in \mathbf C}\coprod}{\mathbf y(C)}\overset{\epsilon}\to P$$
    where $\coprod$ denotes the coproduct and for each object $B$ the maps are defined for each $v:B\to C$ or $C'$ as follows

    $$\epsilon_B(C,p;v)=P(v)p, \theta_B(u,p;v)=(C,p;uv),\tau_B(u,p;v)=(C',pu;v)$$
\end{question}
   \begin{proof}

       Key idea: If there is a surjective map $A\twoheadrightarrow X$, then $X$ can be recovered from $A$ by identifying points in $A$ that is mapped to the same point in $X$.

       By Yoneda's lemma, an element in $P(C)$ is a natural transformation $\y (C) \to P$, and a map $C'\to C$ is a natural tranformation $\y (C')\to \y(C)$. From this point of view, this amounts to prove $P$ is a equalizer:

       $$\underset{\underset{\underset{\y(C)\to P}{\y(C')\to \y(C)}}u}\coprod \mathbf y (C')\overset{\theta}{\underset{\tau}\rightrightarrows}\underset{\y(C)\to P}{\underset{C\in \mathbf C}\coprod}{\mathbf y(C)}\overset{\epsilon}\to P$$

       To make sense of it, we consider affine schemes. The analogue claim in language of schemes is that we can prove every scheme $X$ is a colimit of some affine schemes by proving we have the coequalizer:

       $$\underset{u:V'\to V}\coprod\underset{c:V\to X}\coprod V'\overset{\theta}{\underset{\tau}\rightrightarrows}\underset{a:U\to X}\coprod U\overset{\epsilon}\to X$$

       where the $U,V,V'$'s are affine.

       Let $X$ be the projective space $\Ps^1$, to visualise it, just consider $\Cn\Ps^1$ which is the Riemann sphere. Note that $X$ itself is not affine, but it can be covered by two copies of affine $1$-space $\An^1$. With one copy of $\An^1$ covering everthing in the sphere except for the north pole, and another copy $\widetilde{\An}^1$ covering everthing in the sphere except for the south pole. Also there is an inclusion map from the affine scheme $\Gm$ of multiplicative group to $X$, which covers every point on the sphere except for the poles.

       Just to keep things simple, let $i_N:\An^1\to X$ be the only map we are considering here from $\An^1\to X$, and same for $i_S:\widetilde{\An}^1\to X,i:\Gm\to X$ which are the canonical inclusion maps, so we will have $\coprod_{a:U\to X}U$ has one copy of each of $\An^1,\widetilde{\An}^1,\Gm$. Now we want to recover $X$ from the disjoint union. We have a canonical surjection $\An^1\coprod\widetilde{\An}^1\coprod\Gm\twoheadrightarrow X$, so what we want is to glue the points in $\An^1,\widetilde{\An}^1,\Gm$ which are mapped to the same point.

       We ask when can we get two points in the disjoint union are mapped to the same point in $X$: Analyse the $\underset{u:V'\to V}\coprod\underset{c:V\to X}\coprod V'$. Given an element in it, the element consists the following information: two affine schemes $V$ and $V'$, a map $c:V\to X$, a map $u:V'\to V$ and a point in $V'$. Using these information, it is two ways to construct an element in $\underset{a:U\to X}\coprod U$: We can either take the map $V'\to X$ which is obtained by composing $u$ and $c$, with the point in $V'$, or we can take the map $V\to X$ with the point the image in $V$ under $u$ of the given point in $V'$. And we have a canonical way to get a point in $X$ given a point in $U$ and a map $a:U\to X$, namely sending the give point to $X$ using the map. 

       \begin{figure}[ht]
        \centering
        \includegraphics[width=0.4\linewidth]{C:/Users/surface/Downloads/fig1.jpg}
       \end{figure}
       
       
       In our case, the inclusion $\Gm\to \An^1$, the canonical map $\An^1\to X$ and the point in $\Gm$ as shown in the picture consists an element in $\underset{u:V'\to V}\coprod\underset{c:V\to X}\coprod V'$. If we use the first way described above to construct element in $\underset{a:U\to X}\coprod U$, we will get the pair consists of the same point we start with and the map $i_N\circ j_N:\Gm\to X$. If we use the second way, we will get the point $\An^1$ corresponds to the point we start with together with the map $i_N$. Both of these two elements will be sent to the same point in $X$, so we should identify them in the disjoint union $\An^1\coprod\widetilde{\An}^1\coprod\Gm$. So each pair of elements in $\An^1\coprod\widetilde{\An}^1\coprod\Gm$ comes from applying the different map on the same element should be identified. And after identifying all such elements, we will get $X$.


       This is saying $X$ is a colimit of the diagram on the left part of:

       \begin{figure}[ht]
        \centering
        \includegraphics[width=0.8\linewidth]{C:/Users/surface/Downloads/fig2.jpg}
       \end{figure}
    



       This diagram is a diagram of affine schemes, so we can say $X$ is a colimit of affine schemes. 
       
       
       Note that it is not the canonical way of covering the $X$ using the three spaces, the canonical way is to consider any map from each of these schemes to $X$. For instance, we do not only consider the canonical inclusion $\An^1\to X$, but the map $x\to x^{-1}$ which will cover each point of $X$ except for the south pole (instead of north pole, which is not covered by the canonical inclusion). For the canonical cover, the uncovered point will range over all points in $X$.



       Return to our question, we want to prove $P$ is the equalizer:

       $$\underset{\underset{\underset{\y(C)\to P}{\y(C')\to \y(C)}}u}\coprod \mathbf y (C')\overset{\theta}{\underset{\tau}\rightrightarrows}\underset{\y(C)\to P}{\underset{C\in \mathbf C}\coprod}{\mathbf y(C)}\overset{\epsilon}\to P$$

       By the Yoneda's lemma, $\Hom(\y(C),P)\cong P(C)$ for each $C\in \C$. From the coproduct $\underset{\y(C)\to P}{\underset{C\in \C}\coprod}\y(C)$, we can get the information $\Hom(\y(C),P)\cong P(C)$ for each $C$, in this sense, the coproduct covers $P$. So we can use the surjection $\Hom(\y(C),P)\to P(C)$ and identify the overlapping point together to recover $P$. Hence $P$ is a colimit of representables in the same sense as $X$ is a colimit of affine schemes.

       







      
       
    
   \end{proof}
\end{document}